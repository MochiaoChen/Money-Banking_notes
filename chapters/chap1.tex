% =================================================
% Chapter 1: 货币 (Money)
% =================================================
\chapter{货币 (Money)}\index{货币 (Money)}

\section{导言:为何从货币开始?}

金融 (Finance)\index{金融 (Finance)} 是社会资金融通的总称。在现代经济中,资金主要以货币形式存在,而银行是资金融通的核心渠道。因此,货币银行是整个金融体系的基石,对金融学的研究必须从理解货币开始。正如习近平总书记在阐释金融强国时所强调的,强大的货币是金融强国的根基之一。近年来,无论是黄金价格的持续攀升,还是各国央行对货币政策的积极调整,都凸显了货币在宏观经济调控与微观资产配置中的核心地位。

从宏观层面看,货币是发行者(通常是中央银行)的负债,其数量的多寡直接关系到通货膨胀 (Inflation)\index{通货膨胀 (Inflation)} 和系统性金融风险 (Systemic Financial Risk)\index{系统性金融风险 (Systemic Financial Risk)}。从微观层面看,货币是持有者的资产,是私人部门进行资产组合管理 (Asset Portfolio Management)\index{资产组合管理 (Asset Portfolio Management)} 和风险定价 (Risk Pricing)\index{风险定价 (Risk Pricing)} 的基础。因此,本章将深入探讨货币的本质。

\section{货币的含义 (The Meaning of Money)}

在经济学中,货币的定义比我们日常口语中的“钱”要宽泛。

\begin{definition}[货币 Money]
货币\index{货币 (Money)}(或称货币供给\index{货币供给 (Money Supply)})是在商品或服务支付中或在偿还债务时被普遍接受的任何东西。
\end{definition}

需要注意的是,货币是一个存量概念 (Stock Concept)\index{存量 (Stock)},它与财富和收入这两个概念有本质区别。
\begin{itemize}
    \item \textbf{财富 (Wealth)\index{财富 (Wealth)}:} 指个人或经济体拥有的、能够储存价值的所有资产的总和,它是一个存量概念。货币只是财富的一种形式。
    \item \textbf{收入 (Income)\index{收入 (Income)}:} 指单位时间内赚取的报酬或收益,例如月薪或年薪。它是一个流量概念 (Flow Concept)\index{流量 (Flow)}。
\end{itemize}

\section{货币的职能 (Functions of Money)}

货币之所以为货币,是因为它在经济中扮演着四个关键角色或职能。

\subsection{交易媒介 (Medium of Exchange)\index{货币职能!交易媒介 (Medium of Exchange)}}
这是货币最核心的职能。货币作为交易媒介,极大地提高了经济效率。
\begin{itemize}
    \item 它消除了物物交换经济中“双重欲望巧合 (Double Coincidence of Wants)”\index{双重欲望巧合 (Double Coincidence of Wants)}的难题,从而显著降低了交易成本 (Transaction Costs)\index{交易成本 (Transaction Costs)}。
    \item 促进了专业化分工和市场交换。
\end{itemize}
作为一种有效的交易媒介,货币必须具备以下特征:易于标准化、被广泛接受、可分割、便于携带、不易变质。

\subsection{计价单位 (Unit of Account)\index{货币职能!计价单位 (Unit of Account)}}
货币为经济活动提供了一个统一的价值衡量标准,即所有的商品、服务和资产都可以用货币单位来标价。
\begin{itemize}
    \item 它大大减少了经济中所需的价格数量,简化了交易决策。
    \item 形成了价格体系 (Price System)\index{价格体系 (Price System)},是市场机制有效运行的基础。
\end{itemize}

\subsection{价值贮藏 (Store of Value)\index{货币职能!价值贮藏 (Store of Value)}}
货币可以将今天的购买力储存起来并转移到未来。
\begin{itemize}
    \item 任何资产(如股票、债券、房产)都具有价值贮藏功能,但货币是所有资产中流动性 (Liquidity)\index{流动性 (Liquidity)} 最高的。流动性指一项资产能够以较低的成本迅速转换为交易媒介的特性。
    \item 货币作为价值贮藏手段的主要缺点是,在通货膨胀时期,它的实际价值会下降。
\end{itemize}

\subsection{延期支付标准 (Standard of Deferred Payment)\index{货币职能!延期支付标准 (Standard of Deferred Payment)}}
货币被广泛用于衡量和清偿未来的债务。信贷交易(如贷款、分期付款)的合约通常以货币单位来计价,这使得跨时期的经济活动成为可能。这一职能也扩展了货币的流通范围,但也潜藏着信用危机 (Credit Crisis)\index{信用危机 (Credit Crisis)}的风险。

\begin{examplebox}{思考题:货币职能的识别}
在以下情景中,货币分别体现了哪种职能?
\begin{enumerate}
    \item 布鲁克接受货币作为她日常工作的报酬,因为她知道可以用这些钱购买商品和服务。
    \item 蒂姆想比较橙子和苹果的相对价值,因此他查看了这两种商品以货币单位标出的每磅价格。
    \item 玛丽亚预计她未来的支出会增加,因此决定增加其储蓄账户中的余额。
\end{enumerate}
\textbf{解答:} (1) 交易媒介; (2) 计价单位; (3) 价值贮藏。
\end{examplebox}


\section{支付体系的演变 (Evolution of the Payments System)}

支付体系是指社会用于进行商品和服务交换的方法。它的演变过程反映了货币形式的变迁。

\begin{enumerate}
    \item \textbf{商品货币 (Commodity Money)\index{商品货币 (Commodity Money)}:}
    任何被选作货币的、自身具有内在价值的商品,例如贝壳、牛、盐等。其缺点是价值不稳定、不易分割和携带。

    \item \textbf{金属货币 (Metallic Money)\index{金属货币 (Metallic Money)}:}
    使用具有高内在价值的贵金属(如金、银)作为货币。它比一般商品货币更具优势,但仍存在重量和成色检验不便的问题,直到铸币的出现才得以解决。

    \item \textbf{信用货币 (Fiat Money)\index{信用货币 (Fiat Money)}:}
    也称法定货币 (Legal Tender)\index{法偿货币 (Legal Tender)}。它是由政府法令规定为货币的纸币或硬币,其自身没有内在价值。它的价值来自于人们对其发行者(政府和中央银行)的信任和法律的强制力。

    \item \textbf{支票 (Checks)\index{支票 (Checks)}:}
    支票是向银行发出的、将资金从一个账户转移到另一个账户的指令。它提高了大额交易的效率和安全性,但处理需要时间和成本。

    \item \textbf{电子货币 (Electronic Money)\index{电子货币 (Electronic Money)}:}
    以电子形式存在的货币,通过电子设备进行支付。主要形式包括:
    \begin{itemize}
        \item \textbf{借记卡 (Debit Card):} 允许消费者从其银行账户中直接扣款。
        \item \textbf{电子支付 (E-Payment):} 如支付宝、微信支付等,通过移动应用完成资金转移。
    \end{itemize}

    \item \textbf{数字货币 (Digital Currency)\index{数字货币 (Digital Currency)}:}
    这是一个广义概念,指仅以数字或电子形式存在的货币,主要包括以下三类:
    \begin{itemize}
        \item \textbf{加密货币 (Cryptocurrency)\index{加密货币 (Cryptocurrency)}:} 如比特币 (Bitcoin)\index{比特币 (Bitcoin)},是一种依赖加密技术来验证和记录交易的去中心化数字资产。它运行在一个称为区块链 (Blockchain)\index{区块链 (Blockchain)} 的公共分布式账本上。
        \item \textbf{稳定币 (Stablecoin)\index{稳定币 (Stablecoin)}:} 一种旨在通过与法定货币、商品或其它金融工具挂钩来维持价值稳定的加密货币。
        \item \textbf{中央银行数字货币 (Central Bank Digital Currency, CBDC)\index{中央银行数字货币 (CBDC)}:} 由中央银行发行的数字形式的法定货币,是中央银行的直接负债。中国的数字人民币 (e-CNY)\index{数字人民币 (e-CNY)} 就是一个典型的例子。
    \end{itemize}
\end{enumerate}


\section{货币的计量 (Measuring Money)}

由于不同资产的“货币性”(即流动性)不同,中央银行通常根据流动性的强弱来划分和计量货币供给,形成不同的货币总量指标。

\subsection{美国的货币总量 (The Federal Reserve's Monetary Aggregates)\index{货币总量 (Monetary Aggregates)}}
美国联邦储备系统 (Federal Reserve System)\index{联邦储备系统 (Federal Reserve System)} 主要使用以下两个指标:
\begin{itemize}
    \item \textbf{M1\index{M1}:} 流动性最强的货币。
    \begin{equation*}
        M1 = C + D
    \end{equation*}
    其中,$C$ 代表通货 (Currency),即流通中的现金;$D$ 代表活期存款 (Demand Deposits),包括旅行支票、支票账户存款等。

    \item \textbf{M2\index{M2}:} 在M1的基础上,增加了部分流动性稍差的资产。
    \begin{equation*}
        M2 = M1 + \text{小额定期存款} + \text{储蓄存款和货币市场存款账户} + \text{货币市场共同基金份额}
    \end{equation*}
\end{itemize}
由于金融创新的发展,M1和M2的短期走势可能出现分化,因此政策制定者需要关注不同层次的货币总量。

\subsection{我国的货币层次 (China's Monetary Aggregates)}
中国人民银行 (The People's Bank of China)\index{中国人民银行 (People's Bank of China)} 对货币层次的划分如下:
\begin{itemize}
    \item \textbf{M0\index{M0}:} 流通中现金 (Currency in Circulation),指银行体系以外流通的现金。
    \item \textbf{M1\index{M1}:} 狭义货币 (Narrow Money),通常被视为现实购买力。
    \begin{equation*}
        M1 = M0 + \text{活期存款}
    \end{equation*}
    \textit{注:自2025年1月起,我国M1统计口径修订为:流通中货币(M0)、单位活期存款、个人活期存款、非银行支付机构客户备付金。}
    \item \textbf{M2\index{M2}:} 广义货币 (Broad Money),反映了社会总需求和潜在购买力。
    \begin{equation*}
        M2 = M1 + \text{定期存款} + \text{储蓄存款} + \text{其他存款}
    \end{equation*}
    M2与M1的差额被称为准货币 (Quasi-money)\index{准货币 (Quasi-money)},主要是定期存款和储蓄存款,它们的流动性较差,但经过一定手续后可以变为现实购买力。
\end{itemize}

\section{金属货币制度 (System of Metallic Money)\index{金属货币制度 (System of Metallic Money)}}
货币制度 (Monetary System)\index{货币制度 (Monetary System)} 是国家以法律形式确定的货币流通的结构和组织形式。历史上,金属货币制度,特别是金本位制,曾长期占据主导地位。

\subsection{金本位制 (The Gold Standard)\index{金本位制 (Gold Standard)}}
金本位制是以黄金为本位币的货币制度,存在以下三种主要形式:
\begin{enumerate}
    \item \textbf{金币本位制 (Gold Coin Standard):}\index{金本位制!金币本位制 (Gold Coin Standard)} 这是最典型的金本位制。其特点是:金币可以自由铸造和熔化;银行券等价值符号可以自由兑换成金币;黄金可以自由输入和输出。该制度下,汇率由铸币平价决定,并通过黄金输送点 (Gold Points)\index{黄金输送点 (Gold Points)} 机制维持稳定。
    \item \textbf{金块本位制 (Gold Bullion Standard):}\index{金本位制!金块本位制 (Gold Bullion Standard)} 国内不流通金币,只发行可按规定条件兑换成金块的银行券。这是一种节约黄金的制度。
    \item \textbf{金汇兑本位制 (Gold Exchange Standard):}\index{金本位制!金汇兑本位制 (Gold Exchange Standard)} 本国货币与另一个实行金本位制国家(通常是美国或英国)的货币挂钩,通过在外汇市场上买卖该国货币来维持本币稳定。本国货币不能直接兑换黄金,但可以通过兑换成外汇再间接兑换黄金。
\end{enumerate}

\subsection{布雷顿森林体系 (The Bretton Woods System)\index{布雷顿森林体系 (Bretton Woods System)}}
第二次世界大战后建立的国际货币体系,本质上是一种“双挂钩”的金汇兑本位制。
\begin{itemize}
    \item \textbf{美元与黄金挂钩:} 美国承诺按每盎司35美元的官价兑换黄金。
    \item \textbf{其他国家货币与美元挂钩:} 各国货币确定对美元的固定汇率。
\end{itemize}
该体系的崩溃源于“特里芬两难 (Triffin Dilemma)”\index{特里芬两难 (Triffin Dilemma)}:一方面,世界经济发展需要美国提供更多的美元作为国际清偿手段;另一方面,美元供给的不断增加会动摇人们对美元能兑换黄金的信心,最终导致挤兑。1971年,尼克松总统宣布美元停止兑换黄金,标志着布雷顿森林体系瓦解。

\section{信用货币制度 (System of Fiat Money)\index{信用货币制度 (System of Fiat Money)}}
布雷顿森林体系崩溃后,世界各国普遍进入信用货币制度,其主要特征如下:
\begin{itemize}
    \item \textbf{黄金非货币化:} 货币与黄金完全脱钩,不再规定含金量。黄金回归其商品和金融资产属性。
    \item \textbf{货币供给信用化:} 流通中的货币主要由现金和银行存款构成,由中央银行和商业银行通过信贷业务创造。中央银行通过货币政策对信贷规模进行调控。
    \item \textbf{货币形式多样化:} 货币的范围不断扩大,从现金、活期存款,延伸到各类电子和数字形式。
\end{itemize}
中国的人民币制度就是一种典型的信用货币制度。人民币是中国的法定货币,其发行权集中于中国人民银行,通过信贷程序进行投放和回笼。