\chapter{金融市场 (Financial Markets)}
\label{chap:financial_markets}

金融市场是经济体系中至关重要的一环,它促进了资金从盈余方(储蓄者)向短缺方(投资者)的有效转移。理解金融市场的结构、参与者和工具,是掌握货币银行学理论与实践的基础。本章将系统介绍金融市场的核心组成部分,包括货币市场、债券市场、股票市场以及金融衍生品市场。

% --- 金融体系中的资金流动 ---
\section{金融体系的资金流动与市场结构}
\label{sec:flow_and_structure}

金融体系的核心功能是实现资金融通。资金的流动主要通过两种方式:直接融资和间接融资。

\begin{figure}[htbp]
    \centering
    \includegraphics[width=0.9\textwidth]{images/flow_of_funds.png} % 请将图片路径替换为您的文件路径
    \caption{金融体系中的资金流动 (Flows of Funds Through the Financial System)}
    \label{fig:flow_of_funds}
\end{figure}

如图 \ref{fig:flow_of_funds} 所示,资金总是从\textbf{资金盈余方} (Lender-Savers)\index{资金盈余方 (Lender-Savers)}——如家庭、部分企业和政府——流向\textbf{资金短缺方} (Borrower-Spenders)\index{资金短缺方 (Borrower-Spenders)}——如需要投资的企业、有预算赤字的政府等。

\begin{itemize}
    \item \textbf{直接融资 (Direct Finance)\index{直接融资 (Direct Finance)}}: 资金短缺方通过在金融市场上出售金融工具(如债券和股票)直接从资金盈余方获得融资。
    \item \textbf{间接融资 (Indirect Finance)\index{间接融资 (Indirect Finance)}}: 资金流动需通过\textbf{金融中介机构} (Financial Intermediaries)\index{金融中介机构 (Financial Intermediaries)},如银行、保险公司等。金融中介机构从储蓄者手中吸收资金,再将这些资金贷给需要融资的借款人。
\end{itemize}

\subsection{金融市场的结构 (Structure of Financial Markets)}
\label{subsec:market_structure}
为了更好地理解金融市场,我们可以从不同维度对其进行划分:
\begin{itemize}
    \item \textbf{债权市场与权益市场 (Debt and Equity Markets)}
    \begin{itemize}
        \item \textbf{债权工具 (Debt Instruments)\index{债权工具 (Debt Instruments)}}: 这类工具规定了发行人向持有人在未来特定日期支付固定金额的义务,它们具有\textbf{到期日} (maturity)。例如:债券 (bond)、贷款。
        \item \textbf{权益工具 (Equity Instruments)\index{权益工具 (Equity Instruments)}}: 这类工具代表了对公司净收入和资产的所有权请求权,通常没有到期日。例如:股票 (stocks)。持有权益工具意味着成为公司的所有者之一。
    \end{itemize}
    \item \textbf{货币市场与资本市场 (Money and Capital Markets)}
    \begin{itemize}
        \item \textbf{货币市场 (Money Markets)\index{货币市场 (Money Markets)}}: 交易短期(通常指期限在一年以内)债权工具的市场。其特点是流动性强、风险较低。
        \item \textbf{资本市场 (Capital Markets)\index{资本市场 (Capital Markets)}}: 交易长期(期限在一年以上)债权工具和权益工具的市场。其融资期限长,风险相对较高,是形成资本的重要场所。
    \end{itemize}
\end{itemize}

\section{第一部分:货币市场 (Money Markets)}
\label{sec:money_markets}
\begin{definition}[货币市场]
\label{def:money_market}
货币市场 (Money Markets)\index{货币市场 (Money Markets)} 是指交易期限在一年以内的短期债务投资工具的市场。这些资产通常具有高度的\textbf{安全性} (safety)、强大的\textbf{流动性} (liquidity),可以随时方便地转换为现金,但相应的\textbf{回报率} (rates of return) 也相对较低。
\end{definition}

货币市场的主要参与者是政府、大型企业和银行等机构。它们利用货币市场来满足短期流动性需求或监管要求,同时也为拥有闲置现金的机构提供了赚取利息的渠道。

\subsection{货币市场工具 (Money Market Instruments)}
\label{subsec:money_market_instruments}
货币市场包含多种高度流动性的短期金融工具。
\begin{itemize}
    \item \textbf{国库券 (Treasury Bills)\index{国库券 (Treasury bills)}}: 由中央政府为弥补短期资金缺口而发行的短期债务凭证,期限通常从几天到一年不等。由于有政府信用背书,它被认为是风险最低的货币市场工具之一。

    \item \textbf{可转让定期存单 (Negotiable Certificate of Deposit, NCD)\index{可转让定期存单 (NCD)}}: 由银行发行的大额定期存款凭证(在美国,通常最低面值为10万美元)。与普通定存不同,NCD可以在到期前在二级市场上出售,具有较好的流动性。

    \item \textbf{商业票据 (Commercial Paper)\index{商业票据 (Commercial Paper)}}: 由信用评级高的大型企业发行的无担保短期票据,用于满足短期现金流需求。由于其无担保特性,只有信誉卓著的公司才能成功发行。

    \item \textbf{货币市场共同基金 (Money Market Mutual Funds, MMMF)\index{货币市场共同基金 (MMMF)}}: 投资于一篮子货币市场工具(如国库券、NCD、商业票据)的共同基金。它为个人投资者提供了参与货币市场的渠道,具有高流动性和低风险的特点。

    \item \textbf{回购协议 (Repurchase Agreement, Repo)\index{回购协议 (Repo)}}: 也称为 Repo 或 RP,是一方(通常是资金需求方)出售证券给另一方,并同意在未来某个约定日期以稍高的价格购回该证券的协议。这本质上是一种以证券为\textbf{抵押品} (collateral) 的短期贷款。
    \begin{itemize}
        \item 对于出售证券并在未来回购的一方,该交易是\textbf{回购 (Repo)}。
        \item 对于购买证券并在未来卖出的一方,该交易是\textbf{逆回购 (Reverse Repo, RRP)\index{逆回购协议 (Reverse Repo)}}。
        \item 这些协议的隐含利率被称为\textbf{回购利率 (Repo Rate)\index{回购利率 (Repo Rate)}},通常被视为无风险隔夜利率的代表。
    \end{itemize}
\end{itemize}

\subsection{重要的货币市场利率}
\label{subsec:money_market_rates}

\subsubsection{联邦基金利率 (Federal Funds Rate)}
\textbf{联邦基金市场 (The Federal Funds Market)\index{联邦基金市场 (Federal Funds Market)}} 是美国存款机构之间借贷\textbf{准备金头寸} (reserve balances)\index{准备金头寸 (reserve balances)}的市场。这些贷款通常是无担保的隔夜贷款。

\begin{definition}[联邦基金利率]
\label{def:fed_funds_rate}
联邦基金利率 (Fed Funds Rate)\index{联邦基金利率 (Fed Funds Rate)} 是存款机构之间为借贷准备金而收取的利率。它是美国最重要的利率之一,也是美联储 (Federal Reserve) 货币政策的主要操作目标。
\end{definition}

美联储通过公开市场操作来影响联邦基金市场上的准备金供给,从而引导实际的\textbf{联邦基金有效利率} (Effective Federal Funds Rate) 在其设定的\textbf{目标区间} (Target Range) 内波动。

\subsubsection{伦敦银行同业拆借利率 (LIBOR)}
\textbf{伦敦银行同业拆借利率 (London InterBank Offered Rate, LIBOR)\index{伦敦银行同业拆借利率 (LIBOR)}} 曾是全球最重要的基准利率之一。它是一组主要国际银行相互借贷无担保资金时所报出的平均利率。LIBOR 曾涵盖10种货币和15种期限。然而,由于其报价机制的脆弱性和一系列操纵丑闻,LIBOR 已被逐步淘汰,并由更可靠的替代基准利率(如 SOFR)所取代。

\subsubsection{上海银行间同业拆借利率 (SHIBOR)}
\textbf{上海银行间同业拆借利率 (Shanghai InterBank Offered Rate, SHIBOR)\index{上海银行间同业拆借利率 (SHIBOR)}} 是中国的基准利率体系。它是由信用等级较高的18家报价银行自主报出的人民币同业拆出利率计算确定的算术平均利率。
\begin{itemize}
    \item SHIBOR 是一种\textbf{单利、无担保、批发性}的利率。
    \item 关键在于,SHIBOR 是基于\textbf{报价}计算的利率,而非真实成交利率。
\end{itemize}

\subsubsection{中国的回购市场利率:R007 与 DR007}
与报价性质的SHIBOR不同,中国的回购市场利率是基于真实交易形成的,因此能更准确地反映市场流动性的松紧状况。
\begin{itemize}
    \item \textbf{R007\index{R007}}: 指银行间市场所有参与机构的7天期质押式回购利率的加权平均数。它是一个覆盖范围广泛的综合性指标。
    \item \textbf{DR007\index{DR007}}: 指\textbf{存款类金融机构}之间以\textbf{利率债}为质押品的7天期回购利率的加权平均数。由于限制了交易对手(银行类)和抵押品(高信用等级),DR007剔除了信用风险溢价,被认为能更好地反映\textbf{银行体系的流动性状况},是中国人民银行重点关注的政策参考利率。
\end{itemize}


\subsection{中央银行在货币市场中的角色}
\label{subsec:central_bank_role}
中央银行是货币市场最重要的参与者,它通过货币市场工具执行货币政策。
\begin{itemize}
    \item \textbf{美联储的回购操作}:
    \begin{itemize}
        \item \textbf{回购交易 (Repo)}: 美联储从交易对手处购买证券,向银行体系\textbf{注入流动性} (inject liquidity),暂时增加准备金余额。
        \item \textbf{逆回购交易 (Reverse Repo)}: 美联储向交易对手出售证券,从银行体系\textbf{回笼流动性} (drain liquidity),暂时减少准备金余额。
    \end{itemize}
    \item \textbf{中国人民银行的回购操作}: 在中国的语境下,央行的操作方向恰好相反。
    \begin{itemize}
        \item \textbf{逆回购操作}: 人民银行从一级交易商处购买有价证券,向市场\textbf{投放流动性}。这是常用的流动性供给工具。
        \item \textbf{正回购操作}: 人民银行向一级交易商出售有价证券,从市场\textbf{回笼流动性}。
    \end{itemize}
\end{itemize}

\begin{remark}
    理解中美两国央行操作术语的差异至关重要。美联储的“Repo”是投放流动性,而中国央行的“逆回购”是投放流动性。记忆的关键在于始终从\textbf{交易对手}(商业银行)的角度出发:向央行卖出证券(得到钱)是逆回购;从央行买入证券(付出钱)是正回购。
\end{remark}

\section{第二部分:债券市场 (Bond Markets)}
\label{sec:bond_markets}

债券市场是资本市场的重要组成部分,专门交易各类债权工具。

\subsection{资本市场概述}
\label{subsec:capital_market_overview}

\begin{definition}[资本市场]
\label{def:capital_market}
资本市场 (Capital Market)\index{资本市场 (Capital Market)} 是交易长期(期限一年以上)金融工具的市场,主要包括\textbf{长期债券}和\textbf{股票}。资本市场为政府和企业的长期投资项目提供融资。
\end{definition}

资本市场按功能可分为一级市场和二级市场:
\begin{itemize}
    \item \textbf{一级市场 (Primary Market)\index{一级市场 (Primary Market)}}: 也称\textbf{发行市场} (Issuance Market),是新证券(如新发行的股票或债券)首次出售给初始购买者的市场。投资银行 (Investment Bank) 在此市场中通常扮演\textbf{承销商} (underwriter) 的角色。
    \item \textbf{二级市场 (Secondary Market)\index{二级市场 (Secondary Market)}}: 也称\textbf{流通市场} (Trading Market),是已发行证券进行买卖交易的市场。
    \begin{itemize}
        \item 二级市场为金融工具提供了\textbf{流动性},使投资者可以方便地将证券变现。
        \item 二级市场的价格决定了一级市场新发行证券的价格,起到了\textbf{价格发现} (price discovery) 的功能。
    \end{itemize}
\end{itemize}
二级市场按组织形式可分为:
\begin{itemize}
    \item \textbf{交易所市场 (Exchanges)\index{交易所 (Exchanges)}}: 买卖双方(或其代理人)在集中的物理或电子场所进行交易,如纽约证券交易所 (NYSE)。
    \item \textbf{场外市场 (Over-the-Counter, OTC)\index{场外市场 (OTC)}}: 交易在分散的地理位置通过做市商网络进行。做市商持有证券库存,随时准备向任何人报出买卖价格。纳斯达克 (NASDAQ) 是最著名的OTC市场范例。
\end{itemize}

\subsection{中国债券种类}
\label{subsec:china_bonds}
中国的债券市场品种丰富,可以按照发行主体进行分类。

\subsubsection{政府债券 (Government Bonds)}
由政府发行的债券,通常被认为信用风险最低,因此也被称为\textbf{“金边债券”}。
\begin{enumerate}
    \item \textbf{国债 (Treasury Bonds)\index{国债 (Treasury Bonds)}}: 由中央政府(财政部)发行,具有最高信用等级。因其价格波动主要反映货币的时间价值和供求状况,也被称为\textbf{利率债} (rate bond)\index{利率债 (Rate Bond)}。
    \begin{itemize}
        \item \textbf{国库券 (Treasury Bills)}: 期限在一年以内的短期国债。
        \item \textbf{中长期国债}: 期限在一年以上的国债。
    \end{itemize}
    
    \item \textbf{地方政府债券 (Local Government Bonds)\index{地方政府债券 (Local Government Bonds)}}: 由地方政府发行。
    \begin{itemize}
        \item \textbf{一般债券 (General Obligation Bonds)}: 为没有收益的公益性项目发行,主要以一般公共预算收入偿还。
        \item \textbf{专项债券 (Revenue Bonds)}: 为有一定收益的公益性项目发行,以项目对应的政府性基金或专项收入偿还。
    \end{itemize}
    
    \item \textbf{准国债 (Quasi-sovereign Bonds)}:
    \begin{itemize}
        \item \textbf{中央银行票据 (Central Bank Bill)\index{中央银行票据 (Central Bank Bill)}}: 中央银行为调节市场流动性而向商业银行发行的短期债务凭证,是货币政策工具之一。
        \item \textbf{政府机构支持债券}: 由政府控股或提供信用担保的机构(如原铁道部、中央汇金公司)发行的债券。
    \end{itemize}
\end{enumerate}

\subsubsection{金融债券 (Financial Bonds)\index{金融债券 (Financial Bonds)}}
由银行等金融机构发行的债券。
\begin{enumerate}
    \item \textbf{政策性金融债券 (Policy Financial Bonds)\index{政策性金融债券}} : 由国家开发银行、中国进出口银行、中国农业发展银行这三家政策性银行发行,用于支持国家重大项目和政策性业务。其发行量巨大,流动性好,信用等级仅次于国债。
    \item \textbf{金融机构债券}: 由商业银行、证券公司、保险公司等发行的债券,用于调整负债结构、补充资本金等。
\end{enumerate}

\subsubsection{信用债券 (Credit Bonds)\index{信用债券 (Credit Bonds)}}
由政府之外的主体(主要是企业)发行的债券。与利率债相比,信用债含有\textbf{违约风险} (default risk),因此其收益率通常高于同期限的国债,高出的部分被称为\textbf{信用利差} (credit spread)。
\begin{enumerate}
    \item \textbf{企业债券 (Enterprise Bonds)}: 在中国特指由发改委核准、主要由中央所属企业或国有企业发行的债券,通常与特定项目投资相关,期限较长。
    \item \textbf{公司债券 (Corporate Bonds)\index{公司债券 (Corporate Bonds)}}: 由股份有限公司或有限责任公司发行,发行主体范围更广,资金用途更灵活。
    \item \textbf{可转换公司债券 (Convertible Bonds)}: 持有者可在一定时期内按特定比例或价格将其转换为公司普通股的债券,兼具债权和期权的特性。
    \item \textbf{企业短期融资券 (Short-term Commercial Paper)}: 期限在一年以内的无担保融资债券,在银行间市场发行。
    \item \textbf{中期票据 (Medium-Term Notes, MTN)\index{中期票据 (MTN)}}: 期限通常在1-10年的债务融资工具,在银行间市场发行。
\end{enumerate}

% 本章第一部分内容结束

\section{第三部分:股票市场 (Stock Markets)}
\label{sec:stock_markets}

股票市场是资本市场中最受关注的部分,它为企业提供了重要的股权融资渠道,也为投资者提供了分享企业成长红利的机会。

\subsection{股票的基本概念 (Basic Concepts of Stocks)}
\label{subsec:stock_basics}
\begin{definition}[股票]
\label{def:stock}
股票 (Stocks)\index{股票 (Stock)} 是一种\textbf{权益工具} (Equity Instrument),代表持有者对一家公司\textbf{净收入} (net income) 和\textbf{资产} (assets) 的所有权请求权。持有股票的投资者即成为公司的股东。
\end{definition}

股票主要分为两种类型:
\begin{itemize}
    \item \textbf{普通股 (Common Stock)\index{普通股 (Common Stock)}}: 普通股股东是公司的基本所有者,拥有选举董事会、对公司重大政策进行投票等权利。在公司清算时,他们在债权人、债券持有人和优先股股东之后获得剩余资产的分配权。
    \item \textbf{优先股 (Preferred Stock)\index{优先股 (Preferred Stock)}}: 优先股股东通常没有或只有有限的投票权。但作为补偿,他们在股利分配和公司清算时的资产分配上,享有优先于普通股股东的权利。
\end{itemize}

\subsection{股票估值 (Stock Valuation)}
\label{subsec:stock_valuation}
股票估值是确定股票\textbf{内在价值} (intrinsic value)\index{内在价值 (Intrinsic Value)}的过程。通过比较内在价值与当前市场价格,投资者可以判断股票是被高估还是低估。

\subsubsection{市盈率法 (Price-to-Earnings Ratio, P/E)}
\label{ssubsec:pe_ratio}
\begin{definition}[市盈率]
\label{def:pe_ratio}
市盈率 (Price-to-Earnings Ratio, P/E)\index{市盈率 (P/E Ratio)} 是衡量公司股价相对于其\textbf{每股收益} (Earnings Per Share, EPS)\index{每股收益 (EPS)}的估值指标。其计算公式为:
\begin{equation}
    \text{P/E Ratio} = \frac{\text{每股市价 (Share Price)}}{\text{每股收益 (EPS)}}
\end{equation}
\end{definition}
市盈率可以基于历史数据(\textbf{静态市盈率}或\textbf{滚动市盈率 TTM}\index{滚动市盈率 (TTM)})或未来预测数据(\textbf{动态市盈率})进行计算。
\begin{itemize}
    \item \textbf{高市盈率} 可能意味着:(1) 公司股票价格被市场高估;(2) 投资者对公司未来盈利的高增长抱有强烈预期。
    \item \textbf{低市盈率} 可能意味着:(1) 公司股票价格被市场低估;(2) 市场对公司未来增长前景持悲观态度。
\end{itemize}
不同行业、不同发展阶段的公司,其合理的市盈率水平也大不相同。科技成长型公司的市盈率通常远高于成熟行业的公司。

\subsubsection{股利贴现模型 (Dividend Discount Model, DDM)}
\label{ssubsec:ddm}
\textbf{股利贴现模型 (Dividend Discount Model, DDM)\index{股利贴现模型 (DDM)}} 的核心思想是:股票的当前价格等于其所有未来预期股利支付的\textbf{现值} (present value) 之和。
\begin{itemize}
    \item \textbf{单期估值模型 (The One-Period Valuation Model)}: 假设投资者持有股票一年,获得一次股利后卖出。
    \begin{equation}
        P_0 = \frac{Div_1}{(1+k_e)} + \frac{P_1}{(1+k_e)}
    \end{equation}
    其中,$P_0$ 是当前股价,$Div_1$ 是第一年年末的股利,$P_1$ 是第一年年末的预期卖出价格,$k_e$ 是股权投资的\textbf{要求回报率} (required return on investment in equity)\index{要求回报率 (Required Return)}。
    
    \item \textbf{广义股利估值模型 (The Generalized Dividend Valuation Model)}: 将单期模型扩展至无限期。
    \begin{equation}
        P_0 = \sum_{t=1}^{\infty} \frac{D_t}{(1+k_e)^t}
    \end{equation}
    该模型表明,股票的价格最终仅由其未来所有股利的现值决定。
    
    \item \textbf{戈登增长模型 (The Gordon Growth Model)\index{戈登增长模型 (Gordon Growth Model)}}: 广义模型的一个简化特例,假设股利以一个\textbf{不变的增长率 g} 永久持续增长。
    \begin{equation}
        P_0 = \frac{D_0(1+g)}{k_e - g} = \frac{D_1}{k_e - g}
        \label{eq:gordon_growth}
    \end{equation}
    其中,$D_0$ 是最近一次已支付的股利,$D_1$ 是下一期预期的股利。该模型成立的核心假设是 $k_e > g$。
\end{itemize}

\begin{examplebox}{戈登增长模型应用}
    一支股票刚刚支付了 \$1.50 的股利。预计该股利将以每年 5\% 的速度永久增长。如果投资者的要求回报率为 15\%,那么这支股票今天的公允价值是多少?
    
    \textbf{解}:
    根据公式 \ref{eq:gordon_growth}:
    $D_0 = 1.50$, $g = 0.05$, $k_e = 0.15$.
    $$ P_0 = \frac{1.50 \times (1+0.05)}{0.15 - 0.05} = \frac{1.575}{0.10} = \$15.75 $$
    该股票的公允价值为 \$15.75。
\end{examplebox}

\subsection{股票市场指数 (Stock Market Indexes)}
\label{subsec:stock_indexes}
\textbf{股票市场指数 (Stock Market Index)\index{股票市场指数 (Stock Market Index)}} 是一个选取特定股票样本集合,用以反映整个股票市场或特定行业/板块表现的统计指标。指数的编制方法各异,最常见的加权方式是按\textbf{市值} (market capitalization)\index{市值 (Market Capitalization)}加权。

\begin{itemize}
    \item \textbf{美国主要股指}:
    \begin{itemize}
        \item \textbf{标准普尔500指数 (S\&P 500)\index{标准普尔500指数 (S\&P 500)}}: 包含500家美国顶尖上市公司,按市值加权,被广泛认为是反映美国股市整体表现的最佳指标。
        \item \textbf{道琼斯工业平均指数 (DJIA)\index{道琼斯工业平均指数 (DJIA)}}: 包含30家美国最大、最具影响力的蓝筹股公司,是一个价格加权指数。
        \item \textbf{纳斯达克综合指数 (Nasdaq Composite)\index{纳斯达克综合指数 (Nasdaq Composite)}}: 涵盖在纳斯达克交易所交易的所有股票,以科技股为主,是反映科技行业景气度的重要指标。
    \end{itemize}
    \item \textbf{中国主要股指}:
    \begin{itemize}
        \item \textbf{上证综合指数 (SSE Composite Index)\index{上证综合指数}} : 反映上海证券交易所全部上市股票的综合表现。
        \item \textbf{深证成份指数 (SZSE Component Index)\index{深证成份指数}} : 从深圳证券交易所选取500家最具代表性的公司作为样本股。
        \item \textbf{沪深300指数 (CSI 300 Index)\index{沪深300指数}} : 从沪深两市中选取300只规模大、流动性好的A股作为样本,是反映中国A股市场整体表现的核心指数。
        \item \textbf{创业板指数 (ChiNext Index)\index{创业板指数}} : 反映深圳创业板市场的运行情况,代表创新型、成长型企业。
        \item \textbf{科创50指数 (STAR 50 Index)\index{科创50指数}} : 反映上海科创板中市值大、流动性好的50只证券的整体表现。
        \item \textbf{恒生指数 (Hang Seng Index)\index{恒生指数}} : 反映香港股票市场整体表现的最重要指标。
    \end{itemize}
\end{itemize}

\section{第四部分:金融衍生品 (Financial Derivatives)}
\label{sec:derivatives}

\begin{definition}[金融衍生品]
\label{def:derivatives}
金融衍生品 (Financial Derivatives)\index{金融衍生品 (Derivatives)} 是一种金融合约,其价值取决于\textbf{标的资产} (underlying asset)\index{标的资产 (Underlying Asset)}、资产组合或基准。它们可以在交易所或场外市场进行交易。
\end{definition}

衍生品的主要功能有两个:
\begin{enumerate}
    \item \textbf{对冲风险 (Hedge Risk)\index{对冲 (Hedging)}}: 利用衍生品来降低或转移持有标的资产所面临的价格波动风险。
    \item \textbf{投机 (Speculation)\index{投机 (Speculation)}}: 在承担风险的同时,预期获得相应的回报,即对标的资产未来价格走势进行押注。
\end{enumerate}

\subsection{期货 (Futures)}
\label{subsec:futures}
\begin{definition}[期货]
\label{def:futures}
期货 (Futures)\index{期货 (Futures)} 是一种标准化的法律合约,约定买方在未来某一特定日期和价格购买一项资产,或卖方在未来某一特定日期和价格出售一项资产。
\end{doc:definition}
期货合约在\textbf{期货交易所} (futures exchange)\index{期货交易所 (Futures Exchange)}进行交易,其标的资产可以是实物商品(如原油、大豆)或金融工具(如股指、国债)。中国主要的金融期货品种包括沪深300股指期货、上证50股指期货、中证500/1000股指期货以及5年期和10年期国债期货。

\subsection{期权 (Options)}
\label{subsec:options}
\begin{definition}[期权]
\label{def:options}
期权 (Options)\index{期权 (Options)} 是一种赋予买方在特定日期前以特定价格买入或卖出标的资产的\textbf{权利},但\textbf{没有义务}的合约。
\end{definition}
\begin{itemize}
    \item 买方为获得这一权利,需要向卖方支付一笔费用,称为\textbf{期权费} (premium)\index{期权费 (Premium)}。
    \item 合约中约定的价格称为\textbf{执行价格} (strike price)\index{执行价格 (Strike Price)}。
    \item 持有者执行期权的行为称为\textbf{行权} (exercise)。
\end{itemize}
期权主要分为两种:
\begin{itemize}
    \item \textbf{看涨期权 (Call Option)\index{看涨期权 (Call Option)}}: 赋予持有者在到期日前以执行价格\textbf{买入}标的资产的权利。当投资者预期资产价格将上涨时,会购买看涨期权。
    \item \textbf{看跌期权 (Put Option)\index{看跌期权 (Put Option)}}: 赋予持有者在到期日前以执行价格\textbf{卖出}标的资产的权利。当投资者预期资产价格将下跌时,会购买看跌期权。
\end{itemize}

\section{货币市场与资本市场的比较}
\label{sec:market_comparison}
下表总结了货币市场与资本市场的主要区别:

\begin{table}[htbp]
    \centering
    \caption{货币市场与资本市场的特征比较}
    \label{tab:market_comparison}
    \begin{tabular}{|l|p{5cm}|p{5cm}|}
        \hline
        \textbf{特征} & \textbf{货币市场 (Money Market)} & \textbf{资本市场 (Capital Market)} \\
        \hline
        \textbf{融资期限} & 短期,从隔夜到一年不等。 & 中长期,从一年到数十年。 \\
        \hline
        \textbf{金融工具} & 同业拆借、国库券、回购协议、商业票据等。 & 股票、中长期债券等。 \\
        \hline
        \textbf{流动性与风险} & 风险低,流动性强,可随时变现。 & 风险高,预期回报也高。 \\
        \hline
        \textbf{市场参与者} & 主要是机构投资者:央行、金融机构、大型企业、政府部门。交易额巨大。 & 参与者多元化,包括机构投资者、个人投资者和外国投资者,交易规模各异。 \\
        \hline
    \end{tabular}
\end{table}

\section{本章小结 (Summary)}
\label{sec:chapter_summary}
本章系统地介绍了金融市场的四大组成部分:
\begin{itemize}
    \item \textbf{货币市场}: 讨论了其短期、高流动性的特点,并详细介绍了中国的同业拆借市场和回购市场及其关键利率(SHIBOR, R007, DR007)。
    \item \textbf{债券市场}: 阐述了资本市场的基本特征(一级/二级市场),并系统梳理了中国各类债券的发行主体和特点。
    \item \textbf{股票市场}: 介绍了股票的种类、两种核心估值原理(市盈率法和股利贴现模型)以及中美两国的主要股票指数。
    \item \textbf{金融衍生品}: 简要说明了期货和期权这两种最重要的衍生工具的基本概念和功能。
\end{itemize}

通过本章学习,应能清晰地对比和区分货币市场与资本市场,列举和描述各类市场的金融工具,并掌握债券和普通股的基本估值计算方法。