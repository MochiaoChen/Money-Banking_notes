% ======================
% Preamble file
% ======================

% --- PACKAGES ---
\usepackage[T1]{fontenc}
\usepackage{amsmath, amssymb, amsfonts, amsthm}
\usepackage[most]{tcolorbox}

% ======================
% PREAMBLE (导言区)
% ======================

% 字体/编码(pdfLaTeX 兼容)
\usepackage[utf8]{inputenc} 

\usepackage[T1]{fontenc}
\usepackage{fontspec}
\defaultfontfeatures{Ligatures=TeX} % 让 ---, `` '' 等老式输入正常工作
\setmainfont{TeX Gyre Termes}      % 类 Times,字形全、稳
\setsansfont{TeX Gyre Heros}       % 类 Helvetica
\setmonofont{DejaVu Sans Mono}     % 等宽,包含 “ ” — 等符号
\usepackage{fix-cm}
% 数学
\usepackage{amsmath, amssymb}
\usepackage{enumitem}

% 版心与纸型(保持你给的几何参数)
\usepackage[
  bindingoffset=0.625in,
  left=.5in, right=.5in,
  top=.8125in, bottom=.9375in,
  paperwidth=6.375in, paperheight=9.25in
]{geometry}
% 也可切换到 Letter:
% \usepackage[margin=.75in, paperwidth=8.5in, paperheight=11in]{geometry}
% 目录标题改名
\renewcommand{\contentsname}{Table of Contents}

% 索引
\usepackage{makeidx}
\makeindex

% 图形与标题页需要的 TikZ
\usepackage{tikz}

% 超链接(最后加载较稳妥)
\usepackage[hidelinks]{hyperref}
\hypersetup{
  pdftitle={Microeconomics Notes},
  pdfauthor={Mochiao Chen},
  pdfpagemode=UseOutlines
}

% (可选)更好看的断词与字距
\usepackage{microtype}


% --- THEOREM ENVIRONMENTS ---
\newtheorem{definition}{Definition}[chapter]
\newtheorem{proposition}{Proposition}[chapter]
\theoremstyle{definition}
\newtheorem{example}{Example}[chapter]
\theoremstyle{remark}
\newtheorem{remark}{Remark}[chapter]

% --- CUSTOM BOXES ---
\newtcolorbox{examplebox}[1]{
    colback=gray!5!white,
    colframe=gray!75!black,
    fonttitle=\bfseries,
    title=#1,
    breakable
}

% --- HYPERREF (最后加载) ---
\usepackage{hyperref}
\hypersetup{
    colorlinks=true,
    linkcolor=blue,
    filecolor=magenta,
    urlcolor=cyan,
    pdftitle={Microeconomics Notes},
    pdfpagemode=FullScreen,
}
